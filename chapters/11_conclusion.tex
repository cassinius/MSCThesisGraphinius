\chapter{Conclusion}
\label{ch:conclusion}

In this thesis we have introduced the Graphinius platform for graph-theoretical Machine Learning experiments built on Web technologies, employing in-browser computations as well as visualization, focusing on a community centered approach.

After discussing some theoretical advantages such a platform could offer in the first chapter and delving into more specific descriptions of the theoretical underpinnings of potential application areas, we also took an interest in how such a product could be marketed, what business models would support it, and where the main advantages lie over its foreseeable competitors.

A generic description of the platform characteristics was given, followed by an in-depth survey of the modern web development cycle, its components and supporting infrastructure. Sifting through a wealth of open source alternatives, we finally decided on how to build every component of the Graphinius ecosystem including the base library, visualization, and communication (history) module.

We saw how three different use cases, theoretically tackled in earlier chapters, can be implemented using Graphinius and presented the output of their respective computations.

Following the presentation of some implementation metrics, such as size of the codebase and performance measurements on different graphs, we finally conducted a rather extensive sweep of interesting challenges and the promise emerging technologies hold for future developments of the platform.

The Graphinius development is still in its early phases, with just a groundwork having been laid as of May, 2016. Remembering the first conception of Graphinius only about six months earlier however, the author is remarkably pleased with the progress and dares to take a very optimistic look into the future.