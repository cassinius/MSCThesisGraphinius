\chapter{Architecture of GRAPHINIUS}
\label{ch:graphinius_architecture}

\begin{figure}[ht]
	% \centering
	\hspace*{-0.5cm}
	\includegraphics[width=1.1\textwidth]{figures/Graphinius_Architecture_pdf}
	\caption{Graphinius platform architecture overview}
	\label{fig_graphinius_architecture}
\end{figure}



\section{Graphinius Base}
\label{sect:graphinius_base}

As the name suggests, the Base offers all the functionality necessary to develop graph-based applications on top of it, including basic graph-computations and algorithms as well as visualization. It is therefore logically composed of those two modules, Graphinius JS and Graphinius VIS as well as a mechanism of communication between those two.
	

\section{Graphinius JS}
\label{sect:graphinius_js}		
	
	\subsection{Graph Core}
	\label{ssect:graph_core}
	
		\subsubsection{Edges}
		\label{sssection: core_edges}
		
		\subsubsection{Nodes}
		\label{sssection: core_nodes}
		
		\subsubsection{Graph}
		\label{sssection: core_graph}
		
		\subsubsection{Traversal}
		\label{sssection: core_traveral}
		
		BFS / DFS implementations... [figure: callback-based DFS]
		
		\subsubsection{Degrees}
		\label{sssection: core_degrees}
		
		\subsubsection{Generators}
		\label{sssection: core_}
		
		probability
		per-node degree
	
	
	\subsection{Graph input readers}
	\label{ssect:input_output}
	
		\subsubsection{CSV}
		\label{sssection: io_csv}
		
		
		\begin{figure}[ht]
			\begin{lstlisting}
			A, B, u, C, u, A, d, B, d, D, d
			B, A, u
			C, A, u, A, d
			D, A, d
			\end{lstlisting}
			\caption{Adjacency list including edge direction}
			\label{fig:adj_list_direction}
		\end{figure}
		
		CSV Edge Lists use the simple format of [StartNode, EndNode [,directed]].		
		
		
		\subsubsection{JSON}
		\label{sssection: io_json}
		
		\begin{figure}[ht]
			\centering
			\hspace*{-1.5cm}
			\includegraphics[width=1.2\textwidth]{figures/search_graph_json}
			\caption{Sample graph in the Graphinius JSON format}
			\label{fig:json_input_graph}
			\small Apart from the 'to' node, direction and weight, any node can exhibit an arbitrarily large feature vector containing any type of information (like patient data, word vectors, etc.). Another special sub-object which the input reader is looking for is the 'coords' object, which specifies the coordinates used in the constant layout renderer of the GraphiniusVIS library.
		\end{figure}
	
	
	\subsection{Algorithms}
	\label{ssect:algorithms}
	
		\subsubsection{Clustering}
		\label{sssection: algo_clustering}
		
		\subsubsection{MinSpanTrees}
		\label{sssection: algo_minspan}
		
		\subsubsection{Shortest Paths}
		\label{sssection: algo_shorest_paths}


\section{The Op-Log / History system}
\label{sect:op_log}

	\subsection{Timeline}
	\label{ssect:timeline}
	
	\subsection{History Object}
	\label{ssect:history_object}
	
		
	
	\begin{landscape}
		\begin{figure}[ht]
			\label{fig_history_workflow}
			\centering
			\vspace{-2.0cm}
			%	\hspace*{0cm}
			\includegraphics[width=1.6\textwidth]{figures/History_Workflow_pdf}
			\caption{Graphinius JS <-> VIS communication via Op-Log}
		\end{figure}
	\end{landscape}

	\subsection{Vocabulary}
	\label{ssect:vocabulary}	

	\subsection{Rendering Mechanism}
	\label{ssect:rendering}


\section{Graphinius VIS}
\label{sect:graphinius_vis}

	\subsection{WebGL rendering}
	\label{ssect:webgl_rendering}	

	\subsection{2D/3D Mode}
	\label{ssect:vis_2d3d}
	
	\subsection{Navigation}
	\label{ssect:vis_navigation}
	
	\subsection{Graph Layouts}
	\label{ssect:vis_layouts}	
	
	\subsection{Interaction / Manipulation}
	\label{ssect:vis_interact_manipulate}
	
	\subsection{Timeline rewind / repeat}
	\label{ssect:vis_timeline}	
	
	\begin{landscape}
		\begin{figure}[ht]
			\label{fig_vis_control_flow}
			% \centering
			\hspace*{-1cm}
			\includegraphics[width=1.9\textwidth]{figures/VIS_Control_Flow}
			\caption{Graphinius VIS control flow}
		\end{figure}
	\end{landscape}


\section{Areas of Applications (AoA)}
\label{sect:areas_of_applications}

	\subsection{Online Editor}
	\label{ssect:aoa_editor}
	
	\subsection{Biomedical Applications}
	\label{ssect:aoa_bioapps}
	
	\subsection{SN Anonymization}
	\label{ssect:aoa_anonym}


\section{Platform Services}
\label{sect:platform_services}

	\subsection{Personal Profile}
	\label{ssect:service_profile}
	
	\subsection{Teams}
	\label{ssect:service_teams}
	
	\subsection{Output / Reports}
	\label{ssect:service_output}