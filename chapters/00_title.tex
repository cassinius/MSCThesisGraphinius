%------------------
%\cleardoublepage

\begin{center}

{\large Bernd MALLE, BSc.} \\
{\large Mat.Nr. 0130547} \\
\vspace{1cm}
{\LARGE \textbf
{
GRAPHINIUS\\ A Web based graph exploration and analysis platform\\
}
}
\vspace{2cm}

{\larger
Master's Thesis \\[1ex]
to achieve the university degree of

Master of Science (MSc)

Master's degree programme:\\ 
Software Development and Business Management

}

\end{center}

\vspace{1cm}

\begin{center}
\includegraphics[height=2.5cm]{figures/tuglogo}

\vspace{2cm} % original 3.5

{
% \begin{tabular}{ll}
%\small
Supervisor:\\
Assoc. Prof. Dr. Andreas HOLZINGER\\

Institute for Information Systems and Computer Media\\
Graz University of Technology


% \end{tabular}
}
\vspace{1cm}

{
Graz, May 2, 2016 %\today \\
}

\end{center}




% --- German Title Page is no longer necessary ------------------------------------------------

%\clearpage
%\begin{center}
%This page intentionally left blank
%\end{center}
%\clearpage
% \cleardoublepage
%
%\selectlanguage{austrian}
%
%\begin{center}
%{\larger
%Masterarbeit  \\[1ex]
%}
%
%{\small
%(Diese Arbeit ist in englischer Sprache verfasst) \\
%}
%\vspace{2cm}
%
%
%{\LARGE \textbf
%{
%YOUR GERMAN TITLE \\
%}}
%
%\vspace{2cm}
%{\large YOUR NAME} \\
%\vspace{1cm}
%  , \\
%UNIVERSITY GERMAN \\
%\end{center}
%\vspace{2cm}
%
%\begin{center}
%\includegraphics[height=1.5cm]{diagrams/tug}
%
%
%\vspace{2cm}
{
% \begin{tabular}{ll}
% \small
%  Supervisor: Univ.-Doz.Ing.Mag.rer.nat.Mag.phil.Dr.phil. Andreas HOLZINGER
% \end{tabular}
%}
%\vspace{1cm}
%
%{
%Graz, DATE \\[2cm]
%}
%\end{center}
%
%
\clearpage
\begin{center}
This page intentionally left blank
%This is to ensure smooth page breaks, i.e. that this starts on uneven pages for nicer appearance
\end{center}
\clearpage
% \cleardoublepage

\selectlanguage{english}


\pagestyle{plain}
\pagenumbering{arabic}
\setcounter{page}{3}
%\pagenumbering{Alph}      % for pdf labels

% --- Statutory declaration
% ----------------------------------------------------

\begin{center}
\textbf{STATUTORY DECLARATION}
\end{center}

\noindent
I declare that I have authored this thesis independently, that I have
not used other than the declared sources / resources, and that I have
explicitly marked all material which has been quoted either literally
or by content from the used sources.

\vspace{1cm}
\begin{flushleft}
\begin{tabular}{lclc}
Graz, DATE & {\hspace*{3cm}} & ~\hfill &  \underline{\hspace*{5cm}} \\
~\hfill & {\hspace*{3cm}} & ~\hfill &  YOUR NAME   \\
\end{tabular}
\end{flushleft}
\vspace{2cm}
\clearpage

\clearpage
\begin{center}
This page intentionally left blank
%This is to ensure smooth page breaks, i.e. that this starts on uneven pages for nicer appearance
\end{center}
\clearpage

\section*{Acknowledgements}
% \section*{Danksagung}

First and foremost, I would like to thank my family, friends and colleagues for their enduring mental and emotional support over the past several years while pursuing my studies.

A special thank you also goes to my supervisor Prof. Andreas Holzinger, who has not flinched when I changed the subject of my Master Thesis three times over the past two and a half years.

Finally, I need to thank many students and professors I met while studying economics at KF Uni Graz during the early 2000s. Without their continuous deterrent this Master thesis in Software Development could never have come into existence.


\clearpage
\begin{center}
This page intentionally left blank
\end{center}
\clearpage

% --- English Abstract ----------------------------------------------------
\section*{Abstract}

Graphs are a fundamental tool of mathematics and can be applied to a very diverse field of modern scientific areas: network routing, social network \& community analysis, image processing, even Anonymization of patient data or fraud detection via belief propagation networks.

A relatively novel addition to that spectrum is the emerging field of computational biology, in which we can find protein-protein interaction networks, metabolics, or connectome graphs. Since Biologists, Medical professionals or Privacy researchers are usually no tech experts, an intuitive, GUI-based research platform could facilitate rapid experimental iterations. Graphinius aims to be such a Web-based, graph theoretical platform offering real-time in-browser computations as well as tightly integrated visualization, interaction, and manipulation of graphs.

In this thesis I will mainly introduce GraphiniusJS and its underlying design principles; however, I am in the lucky position of already having guided the development of a graph visualization library called GraphiniusVIS in the context of a colleague's Master's Project.

Aside from presenting some real-world use cases I will also provide an outlook on the whole, emerging Graphinius platform and its exciting capabilities for research and education.


% --- English Keywords ----------------------------------------------------

% \vspace*{5cm}
\vfill
\noindent
\textbf{Keywords}\\
Web based research platform, graph theory, graph visualization, graph mining, machine learning, data engineering, ML metrics, ML heuristics, algorithmic pipelines, WebGL, interactive Machine Learning \\
\\
\textbf{ÖSTAT classification}\\
1140 Software-Engineering\\
\\
\textbf{ACM classification}\\
Software infrastructure

\clearpage
\begin{center}
This page intentionally left blank
\end{center}
\clearpage
% \cleardoublepage

% --- German Abstract ----------------------------------------------------

\selectlanguage{austrian}

\section*{Kurzfassung}

Graphen sind ein fundamentales, mathematisches Werkzeug und können in vielen wissenschaftlichen Bereichen eingesetzt werden: Netzwerk routing, Soziale Netzwerke, Bildverarbeitung sowie Anonymisierung von Patientendaten oder Betrugsanalyse sind nur einige Beispiele.

Ein relativ neues Betätigungsfeld ist jenes der mathematischen Biologie, in welcher wir Protein-Protein Interaktions-Netzwerke oder metabolische bzw. neuronale Graphen vorfinden. Da Biologen, Ärzte und Datenschützer für gewöhnlich keine Experten in Programmierung sind, würde diesen eine GUI basierte Platform zur intuitiven Berechnung von Graphen entgegenkommen. Dieses Ziel verfolgt Graphinius, indem es eine Browser basierte Graphberechnungs, -interaktions sowie -visualisierungs Plattform bereitstellt.

In dieser Diplomarbeit lege ich hauptsächlich die Kernbibliothek GraphiniusJS und deren Designprinzipien dar; darüberhinaus hatte ich bereits die Ehre, ein Masterprojekt zur Entwicklung einer integrierten Visualisierungsbibliothek names GraphiniusVIS mitzubetreuen.

Nach der Präsentation dreier realistischer Anwendungsfälle und deren Ergebnisse, bildet ein Ausblick auf zukünftige Technologien und Potentiale für die weitere Entwicklung von Graphinius den Abschluss dieser Arbeit.


% --- German Schlüsselwörter ----------------------------------------------------

% \vspace*{5cm}
\vfill
\noindent
\textbf{Schlüsselwörter}\\
Webbasierte Forschungsplattform, Graphentheorie, Graphenvisualisierung, Graph-Auswertung, Maschinelles Lernen, Dateninfrastrukturen, ML Metriken, ML Heuristiken, Algorithmische Sequenzen, WebGL, interaktives Machinelles Lernen\\
\\
\textbf{ÖSTAT Klassifikation}\\
1140 Software-Engineering\\
\\
\textbf{ACM Klassifikation}\\
Software infrastructure

\clearpage
\begin{center}
This page intentionally left blank
\end{center}
\clearpage
% \cleardoublepage

\selectlanguage{english}

% \begin{center}
% \textbf{EIDESSTATTLICHE ERKL\"ARUNG}
% \end{center}
%
% \noindent
% Ich erkl\"are an Eides statt, dass ich die vorliegende Arbeit selbstst\"andig verfasst, andere als die angegebenen Quellen/Hilfsmittel nicht benutzt, und die den benutzten Quellen w\"ortlich und inhaltlich entnommene Stellen als solche kenntlich gemacht habe.
%
%
% \vspace{1cm}
%
% \begin{flushleft}
% \begin{tabular}{lclc}
% Graz, DATE & {\hspace*{3cm}} & ~\hfill &  \underline{\hspace*{5cm}} \\
% ~\hfill & {\hspace*{3cm}} & ~\hfill &  YOUR NAME   \\
% \end{tabular}
% \end{flushleft}
%
%
% \vspace{2cm}

% \cleardoublepage

% \textbf{Abbreviations}
%
% \clearpage
